%% start of file `template.tex'.
%% Copyright 2006-2013 Xavier Danaux (xdanaux@gmail.com).
%
% This work may be distributed and/or modified under the
% conditions of the LaTeX Project Public License version 1.3c,
% available at http://www.latex-project.org/lppl/.
%Version for spanish users, by dgarhdez

\documentclass[11pt,a4paper,roman]{moderncv}        % possible options include font size ('10pt', '11pt' and '12pt'), paper size ('a4paper', 'letterpaper', 'a5paper', 'legalpaper', 'executivepaper' and 'landscape') and font family ('sans' and 'roman')
\usepackage[spanish,es-lcroman]{babel}
\usepackage[utf8]{inputenc}
\usepackage[T1]{fontenc}
\usepackage[english]{babel}
%\usepackage[document]{ragged2e}
\usepackage[document]{ragged2e}
% moderncv themes
\moderncvstyle{classic}                            % style options are 'casual' (default), 'classic', 'oldstyle' and 'banking'
\moderncvcolor{green}                              % color options 'blue' (default), 'orange', 'green', 'red', 'purple', 'grey' and 'black'
%\renewcommand{\familydefault}{\sfdefault}         % to set the default font; use '\sfdefault' for the default sans serif font, '\rmdefault' for the default roman one, or any tex font name
%\nopagenumbers{}                                  % uncomment to suppress automatic page numbering for CVs longer than one page

% character encoding
\usepackage[utf8]{inputenc}                       % if you are not using xelatex ou lualatex, replace by the encoding you are using
%\usepackage{CJKutf8}                              % if you need to use CJK to typeset your resume in Chinese, Japanese or Korean

% adjust the page margins
\usepackage[scale=0.75]{geometry}
%\setlength{\hintscolumnwidth}{3cm}                % if you want to change the width of the column with the dates
%\setlength{\makecvtitlenamewidth}{10cm}           % for the 'classic' style, if you want to force the width allocated to your name and avoid line breaks. be careful though, the length is normally calculated to avoid any overlap with your personal info; use this at your own typographical risks...

% personal data
\name{Christian Elvio}{Quispe Cuba}
\title{Resumé title}                               % optional, remove / comment the line if not wanted
\address{Jr Ucayali 211, 05002}{San Juan Bautista, Ayacucho}{Peru}% optional, remove / comment the line if not wanted; the "postcode city" and and "country" arguments can be omitted or provided empty
\phone[mobile]{+51-937-099-896}                   % optional, remove / comment the line if not wanted
%\phone[fixed]{+2~(345)~678~901}                    % optional, remove / comment the line if not wanted
%\phone[fax]{+3~(456)~789~012}                      % optional, remove / comment the line if not wanted
\email{christian.quispe.26@unsch.edu.pe}                               % optional, remove / comment the line if not wanted
%\homepage{www.johndoe.com}                         % optional, remove / comment the line if not wanted
%\extrainfo{additional information}                 % optional, remove / comment the line if not wanted
%\photo[64pt][0.4pt]{picture}                       % optional, remove / comment the line if not wanted; '64pt' is the height the picture must be resized to, 0.4pt is the thickness of the frame around it (put it to 0pt for no frame) and 'picture' is the name of the picture file
%\quote{Some quote}                                 % optional, remove / comment the line if not wanted

% to show numerical labels in the bibliography (default is to show no labels); only useful if you make citations in your resume
%\makeatletter
%\renewcommand*{\bibliographyitemlabel}{\@biblabel{\arabic{enumiv}}}
%\makeatother
%\renewcommand*{\bibliographyitemlabel}{[\arabic{enumiv}]}% CONSIDER REPLACING THE ABOVE BY THIS

% bibliography with mutiple entries
%\usepackage{multibib}
%\newcites{book,misc}{{Books},{Others}}
%----------------------------------------------------------------------------------
%            content
%----------------------------------------------------------------------------------
\begin{document}
%-----       letter       ---------------------------------------------------------
% recipient data
\recipient{Dear Gentle Being}{Member of the scholarship selection team}
\date{February 18, 2021} %poner en las llaves \today
\opening{Of my consideration,}
\closing{Thank you very much for your time and interest and receive a warm greeting.}
\enclosure[Adjunto]{CV}          % use an optional argument to use a string other than "Enclosure", or redefine \enclname
\makelettertitle
\justify
I introduce myself, I am a recent graduate of physical and mathematical sciences, currently working as a research assistant in a competitive project on ``Effects of aerosols in the air of Ayacucho (Peru) and their correlation with cases of COVID-19'', under direction by Drs Juan Dávalos Prado (principal investigator of the Higher Council for Scientific Research CSIC) and Richard Medina Calderon (Researcher in numerical modeling of aerosol particles in the atmosphere of the University of Texas), said project won is thanks to the Camisea Socioeconomic Development Fund (FOCAM) in collaboration with the university where I was trained.

From an early age my interest in science awakened and was motivated as I grew up by the influence of my brother (currently obtaining a doctorate in applied mathematics at UNICAMP-Brazil), which is why I understood the enormous relevance of the formality of mathematics in research, consequently it led me to be a physicist and participate in scientific activities, presentations and conferences; In this way, it strengthened my curiosity in seeking my identity as a researcher, feeling an enormous affinity for programming, simulation and prediction of reality events, this led me to contact Dr. Juan who graciously agreed to advise me on my thesis project, reaching enlighten me about the capacity of theoretical chemistry and computational modeling, was the precise moment in which I understood what I wanted to dedicate myself to in the future, which is to simulate and predict atmospheric chemistry events followed by modeling and characterizing materials.

It is essential to build professional networks and maintain collaboration, due to my familiarity in air quality modeling, I prioritize my application to the University of Perugia within the consortium of the supercomputer CINECA (world reference of calculation) for studies of “Analysis and measurement of spectroscopy of Differential Optical Absorption (DOAS)” within the framework of atmospheric chemistry under the direction or suggestions of Dr. Alfonso Saiz-Lopez (senior scientist at the CSIC expert in atmospheric chemistry) and Dr. David Michele Cappelletti (Professor at the University of Perugia , expert in chemical technology and air quality). Followed by the University of Barcelona for my interest in materials and I would love to collaborate with Francesc Illas Riera's group responsible for the Laboratory of Computational Materials Science under supercomputer calculations to study ``excited states in specific defects of suspended materials''. Afterwards, I prefer to go to the Autonomous University of Madrid to study ``Cost alternative to the functional Kohn-Sham density theory and search for ab Initio performance'' preferably under the guidance of Dr. José Manuel García de la Vega In addition, in the city of Madrid, the center of the Rocasolano Physical Chemistry Institute is very close, and I have a letter of acceptance of stay at the laboratory, and if time allows me, I could escalate or deepen the experimental part of my thesis and train in instrumentation DOAS in atmospheric chemistry with Dr. Juan and Dr. Alfonso Saiz-Lopez. Finally, the University of Valencia, would like to be part of one of the groups of Ana Cros Stötter and her guide in the research line of the Nanostructured Materials Unit (GMN) to carry out studies of ``Detection and capture of volatile organic compounds (VOCs)''.

The TCCM master's degree curriculum is high in physics, which is why my affinity, solid training, and consistent academic performance make me a consistent candidate for this program. During undergraduate I have belonged within the fifth and upper third, I have collaborated with researchers. While doing my undergraduate thesis, I have been trained under the advice of Dr. Juan, where I have learned a lot about theoretical chemistry and computational modeling, I learned Trueno and Ladom bash scripting cluster coding , Gaussian 09, GaussView 0.5 and Spartan 14 programs.

We recently formed a group of different universities within the country with experience in computational modeling in order to share knowledge obtained from a multidisciplinary network of undergraduate and graduate students interested in outlining their thesis project with us. Our objective is to lead the field of theoretical chemistry and computational modeling in Peru from different parts of the country and abroad, we intend to remain active by publishing scientific articles, presenting posters and conferences, increasing our network with the intention of attracting more interested parties for a symbiotic collaboration.

This scholarship will give me a great opportunity to take the next step in my goals. I consider that my academic and professional experience related to physics; followed by applied research make me a promising candidate. I will be happy to provide more information or documents if necessary. I eagerly await your reply.



\vspace{0.5cm}


\makeletterclosing

\end{document}


%% end of file `template.tex'.
